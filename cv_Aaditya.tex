\documentclass[margin,line]{res}
\usepackage[hidelinks]{hyperref}
\usepackage{color}
\usepackage{footmisc}
\usepackage{amssymb}
\oddsidemargin -.5in
\evensidemargin -.5in
\textwidth=6.0in
\itemsep=0in
\parsep=0in
% if using pdflatex:
%\setlength{\pdfpagewidth}{\paperwidth}
%\setlength{\pdfpageheight}{\paperheight} 

\renewcommand{\thefootnote}{\fnsymbol{footnote}}

\newenvironment{list1}{
  \begin{list}{\ding{113}}{%
      \setlength{\itemsep}{0in}
      \setlength{\parsep}{0in} \setlength{\parskip}{0in}
      \setlength{\topsep}{0in} \setlength{\partopsep}{0in} 
      \setlength{\leftmargin}{0.17in}}}{\end{list}}
\newenvironment{list2}{
  \begin{list}{$\bullet$}{%
      \setlength{\itemsep}{0in}
      \setlength{\parsep}{0in} \setlength{\parskip}{0in}
      \setlength{\topsep}{0in} \setlength{\partopsep}{0in} 
      \setlength{\leftmargin}{0.2in}}}{\end{list}}


\begin{document}

\name{Ankit Kumar Pal (Aaditya)\vspace*{.1in}}

\begin{resume}
\section{\sc Contact Information}
\vspace{.05in}
\begin{tabular}{@{}p{3in}p{4in}}
\href{https://aadityaura.github.io}{\bf aadityaura.github.io} & {\it E-mail:}  \href{mailto:aadityaura@gmail.com }{aadityaura@gmail.com } \\
 & {\it Links:} \href{https://scholar.google.com/citations?user=MXqo0_UAAAAJ&hl=en}{\color{blue}{ Google Scholar}, \href{https://github.com/monk1337/} {Github}, \href{https://www.linkedin.com/in/aadityaura/}{LinkedIn}}\\
\end{tabular}


\section{\sc Research Interests}
Representation Learning on Graphs, Generative Large Language Models (LLMs), and their applications in Healthcare data, Federated learning, ASR \& Audio Analysis

\section{\sc Education}
{\bf Babu Banarasi Das University,}, Lucknow, India \hfill {\bf {May 2017}}\\
%\vspace*{-.3cm}
{\em Bachelor of Technology, Computer Science Engineering }
\vspace*{.3cm}
\begin{list2}
\item {\bf Thesis}: Generative Modeling of Music Sequences with LSTM-based RNN Architecture

\end{list2}
{\bf Anandi Devi S.V.M, Sitapur, India} (ADSVM), Sitapur, India \hfill {\bf {April 2013}}\\
%\vspace*{-.3cm}
{\em 12th - Board of High School and Intermediate Education U.P }
\vspace*{.3cm}
\begin{list2}
\item {\bf Major}: Physics, Chemistry and Mathematics
\end{list2}

\section{\sc Experience}
{\bf Saama Technologies} \hfill {\bf May 2018 - Present}\\
%\vspace{-.3cm}
{\em Senior ML Research Engineer}\hfill{}
\vspace{.2cm}\\
{\textbf{Objective:} {\em Develop Deep Learning/NLP methods and pipelines for clinical data, Lead research projects, and published findings in top ML conferences}
\vspace{.3cm}
\begin{list2}
	\item \textbf{Adverse Event Prediction}: {\em FDA Adverse Event Reporting System (FAERS)} Developed an RNN-LSTM model with Context-Aware Attention to extract pharmacological semantics from clinical notes, achieving 98\% F1 score. Optimized character and word embeddings to enrich contextual representation. Enabled automated adverse event detection across 1M records.
        \vspace{.2cm}
        \item \textbf{Trial Plan Optimizer (TPO):} Designed an ML model using Genentech's clinical trial data to predict site enrollment. Implemented a Python \& Scala AutoML framework with TransmogrifAI. Utilized Categorical Embeddings and tree-based algorithms like XGBoost, LightGBM, and Random Forest to optimize predictions.
        \vspace{.2cm}
        \item \textbf{Unsupervised Medical Monitoring:}
        Conducted analysis of clinical trial data across SDTM domains to identify patient outliers. Leveraged historical patient data and unsupervised models like Autoencoders, Clustering(e.g. K-Means, DBSCAN), Isolation Forest, and One-Class SVM to optimize outlier detection.

        \vspace{.2cm}
        \item \textbf{DeepMap ML Framework (SDTM Automap):} Developed an ML system to automatically generate CDISC SDTM mappings, incorporating Generative Adversarial Networks, Bidirectional LSTM with PubMed and BERT embeddings, and a 3-layer ELMo architecture for multi-task learning across clinical domains, achieving an average accuracy of 95\% in mapping source raw data to SDTM standards.
        \vspace{.2cm}
        \item \textbf{Pharma Graph:} {\em Predictive Modeling of Drug Interactions using Graph Convolutional Networks }
        Built a NER model to extract pharmacological relationships from clinical text. Developed a Graph Convolutional Neural Network with attention mechanisms to model drugs as nodes and their interactions as edges, characterizing consequential effects caused by drug pair interactions.
        \vspace{.2cm}
        \item \textbf{Large Language Models for Healthcare Domain}
        Extracted clinical insights from raw medical documents and PDFs using Retrieval-Augmented Generation, fine-tuned open-source LLMs (e.g., Llama-2, Falcon) using custom instruct-datasets for internal use cases, Developed a Python library for prompt versioning and structured outputs, Generating protocol documents from minimal inputs, and Conducting Research to mitigate LLM hallucinations in the medical domain.
\end{list2}

{\bf Prescience Decision Solutions, Bengaluru, India} \hfill {\bf Feb 2018 - May 2018}\\
%\vspace{-.4cm}
{\em Deep Learning Engineer}\hfill{}
\vspace{.2cm}\\
{\textbf{Objective:} {\em Building a Multidimensional Deep Learning Model to Predict the Bitcoin Price}
\vspace{.3cm}
\begin{list2}

\item Worked on transfer learning, attention methods, and custom POS-Tag embeddings.
\item Created an unofficial Twitter API to get Bitcoin tweets and used it to do LSTM sentiment analysis.
\item Added the sentiment analysis as a feature layer in the main model to improve understanding of the data.
\item Deployed the code \& APIs and built a Chat UI on top of it to interact with the model.
\end{list2}

{\bf Fliptango Global Solutions, Kerala, India}\hfill {\bf Dec 2017 – Feb 2018}\\
%\vspace{-.3cm}
{\em Machine Learning Intern}\hfill{}
\vspace{.2cm}\\
{\textbf{Objective:} {\em Design and implement an ML-driven e-commerce chatbot to optimize user interactions and enhance product recommendations}
\vspace{.3cm}
\begin{list2}
\item Used TensorFlow to leverage transfer learning and optimize models for specific tasks.
\item Added new Commonsense Embeddings from ConceptNet Numberbatch to improve understanding of language.
\item Followed BiLSTM-CNN-CRF paper closely to build a named entity recognition model in TensorFlow.
Achieved 95\% accuracy in the NER model, which was great for pulling out the key entities from user chats.
\end{list2}

\section{\sc Selected Publications}
\textbf{Ankit Pal}, Muru Selvakumar, Malaikannan Sankarasubbu. Multi-label Text Classification using Attention-based Graph Neural Network. In Proc. {\it \textbf{ICAART, '20}.} \href{https://arxiv.org/pdf/2003.11644.pdf}{\color{blue}[Link]}

\textbf{Ankit Pal}, Malaikannan Sankarasubbu. Pay attention to the cough: Early diagnosis of COVID-19 using interpretable symptoms embeddings with cough sound signal processing. In {\it \textbf{ACM '21}.} \href{https://dl.acm.org/doi/10.1145/3412841.3441943}{\color{blue}[Link]}

\textbf{Ankit Pal.} CLIFT: Analysing Natural Distribution Shift on Question Answering Models in Clinical Domain. Poster in {\it \textbf{NeurIPS, '22}.} \href{https://nips.cc/virtual/2022/58229}{\color{blue}[Link]}

\textbf{Ankit Pal}, Logesh Kumar Umapathi and Malaikannan Sankarasubbu.  MedMCQA: A Large-scale Multi-Subject Multi-Choice Dataset for Medical domain Question Answering. In Proc. {\it\textbf{PMLR '22.} } \href{https://proceedings.mlr.press/v174/pal22a/pal22a.pdf}{\color{blue}[Link]}

Madhura Josh\textsuperscript{*}, \textbf{Ankit Pal\textsuperscript{*}}, and Malaikannan Sankarasubbu.  Federated learning for healthcare domain - pipeline, applications and challenges. In {\it \textbf{ACM '22.} } \href{https://dl.acm.org/doi/10.1145/3533708}{\color{blue}[Link]}.
\footnotetext{*equal contribution}

\textbf{Ankit Pal.}  DeepParliament: A Legal domain Benchmark \& Dataset for Parliament Bills Prediction. In Proc. {\it \textbf{EMNLP '22.} } \href{https://aclanthology.org/2022.umios-1.8.pdf}{\color{blue}[Link]}

\textbf{Ankit Pal}, Logesh Kumar Umapathi and Malaikannan Sankarasubbu.  Med-HALT: Medical Domain Hallucination Test for Large Language Models. In Proc. {\it \textbf{EMNLP Conll '23.}} \href{https://arxiv.org/pdf/2307.15343.pdf}{\color{blue}[Link]}

\section{\sc Service} Reviewed Papers for Springer Nature 2021, IEEE Access 2021, IEEE Access 2022


\section{\sc Technical Skills} 
\begin{list2}

\item {\bf Programming:} Python, C language, Scala, Rust
\item {\bf Mobile and Web Technologies:} HTML, CSS, JavaScript
\item {\bf Cloud platforms:} Amazon web services, Google Cloud Platform, and Microsoft Azure
\item {\bf Tools:} Jax, Tensorflow, PyTorch, Keras, Scipy, Pandas, Numpy, LaTeX
\end{list2}}

\section{\sc Teaching Experience} 
\begin{list2}

\item {\bf Shala by IIT Bombay:} DL PI-2\\
Graph Convolutional Networks for NLP \& Knowledge graphs
\end{list2}


\section{\sc ML Projects}
{\textbf{Covid-19 Question-Answering Bot} [2020] {\em }
\vspace{.3cm}
\begin{list2}

\item Extracted keywords and retrieved relevant passages using vector search.
\item Ranked top 5 passages for relevance, selecting the top one.
\item Summarized chosen passage using the BART model
\item Developed APIs and deployed the solution through a Telegram bot.
\end{list2}}

{\textbf{Image \& Product Similarity in E-commerce} [2018] {\em }
\vspace{.3cm}
\begin{list2}

\item Transformed product pages into graphs for structural comparison.
\item Applied graph isomorphism techniques to identify product similarities.
\item Leveraged image vectors to ascertain visual similarity between products.
\item Enhanced product recommendation accuracy through combined structural and visual analysis.
\end{list2}}

{\textbf{Music Generation with LSTM \& Double Stacked GRU} [2017] {\em }
\vspace{.3cm}
\begin{list2}

\item Transformed MIDI files into encoded matrices for processing.
\item Trained both single-layer and double-stacked layer models using LSTM and GRU for music generation.
\end{list2}}

{\textbf{Voice-Controlled Robotic Arm} [2016] {\em }
\vspace{.3cm}
\begin{list2}

\item Constructed a robotic arm with servos, operated by Raspberry Pi on Puppy Linux.
\item Integrated a text-to-speech module to translate vocal commands into actionable tasks.
\item Enabled the robot to execute diverse actions, like grasping a cup and lifting a ball.
\item Secured the second prize in a college technical exhibition for innovation.
\end{list2}}

\section{\sc Talks}
{\textbf {MLOps: The Keystone of Sustainable AI}, Coimbatore, India \hfill {Jan, 2023}}\\
%\vspace*{-.3cm}
{\em Gradient Optimizers Meetup }
\vspace*{.3cm}\\
{\textbf {Federated Learning \& Distributional Shift in Healthcare}, Chennai, India \hfill {Dec, 2022}}\\
%\vspace*{-.3cm}
{\em Gradient Optimizers Meetup }
\vspace*{.3cm}\\
{\textbf {AI in Law: A New Legal Era}, Kangra, India \hfill  {Oct, 2021}}\\
%\vspace*{-.3cm}
{\em District Court Kangra }
\vspace*{.3cm}\\
{\textbf {Reasoning in LLMs Through Math Word Problems}, Chennai, India \hfill {Oct, 2020}}\\
%\vspace*{-.3cm}
{\em ML Researchers Meetup }
\vspace*{.3cm}\\
{\textbf {Graphs Neural Networks for NLP}, IITB, India \hfill  Jul, 2020\\
%\vspace*{-.3cm}
{\em Indian Institute of Technology Bombay, Shala }}
\vspace*{.3cm}\\


\section{\sc Featured Open-Source Projects}
{\textbf{Promptify}} \hfill {\bf{Jan, 2023}}\\
{\em Python and JavaScript} \hfill {\href{https://github.com/promptslab/Promptify/}{\color{blue}{\textbf{[Github]}}}}\\
%\vspace*{-.3cm}
\begin{list2}
\item A module for prompt engineering and versioning, Enabling users to efficiently utilize the GPT and similar prompt-based models \\ to get structured output for various NLP tasks, including NER, QA, Classification, etc 
\item Github Trending repository
\end{list2}}
% \vspace*{.9cm}

{\textbf{Research Papers Search (Resp)}} \hfill {\bf{Jul 15, 2022}}\\
{\em Python} \hfill {\href{https://github.com/monk1337/resp}{\color{blue}{\textbf{[Github]}}}}\\

%\vspace*{-.3cm}
\begin{list2}
\item A module to Retrieves paper citations from Google Scholar
\item Fetches relevant papers by keywords across sources like ACL, ACM, PMLR, etc.
\end{list2}

{\textbf{Mix Data types clustering (Mixclu)} \hfill {\bf{Jan 29, 2022}}\\
{\em Python} \hfill {\href{https://github.com/monk1337/Mixclu/tree/main}{\color{blue}{\textbf{[Github]}}}}\\

\begin{list2}
\item Mixclu is an open-source Python package for doing unsupervised mix data types clustering.
\item Includes a variety of combination methods including kmeans-onehot, gower distance, umap, etc.
\end{list2}

{\textbf{Cough Signal Processing (CSP)} \hfill {\bf{June, 2020}}}}\\
{\em Python} \hfill {\href{https://github.com/coughresearch/Cough-signal-processing}{\color{blue}{\textbf{[Github]}}}}\\

%\vspace*{-.3cm}
\begin{list2}
\item Extracts cough features including spectrograms, contiguous segments, and cough events, etc.
\item Implements various ML and DL algorithms for respiratory audio analysis tasks including automated cough classification, clustering, anomaly detection, etc.
\end{list2}}

{\textbf{Unsupervised Toolbox Library (Unbox)} \hfill {\bf{Mar 18, 2020}}}\\
{\em Python} \hfill {\href{https://github.com/monk1337/Unbox/tree/master}{\color{blue}{\textbf{[Github]}}}}\\

\begin{list2}
\item Implements unsupervised NLG methods, focusing on context-aware semantics embeddings.
\item Workflow includes question generation/answering, deep clustering, and latent representations etc
\end{list2}

{\textbf{Multi-label classification Package (MultiLab)} \hfill {\bf {Nov 26, 2019}}}\\
{\em Python} \hfill {\href{https://github.com/monk1337/MultiLab}{\color{blue}{\textbf{[Github]}}}}\\

%\vspace*{-.3cm}
\begin{list2}

\item Implements classical and state-of-the-art models for multi-label classification
\item Provides dataset preprocessing, loading, and evaluation metrics
\end{list2}

\end{resume}
\end{document}